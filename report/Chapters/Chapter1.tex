\chapter{Introduction}
Personal identification is becoming increasingly important in our world today where automated systems are used to control access to facilities which affect our personal and financial security. Examples are internet banking and access control to buildings respectively. Biometrics is the science of identifying individuals based on personal attributes \cite{jain2007handbook}. It has the advantage that individuals can be identified via personal characteristics, as opposed to what they might possess (an identification card) or what they may remember (a password) \cite{Banescu2010}.
\par
A novel, and fairly recent biometric technique \cite{Wang2011}, is the identification of individuals based on the vascular patterns of their hands or fingers. This project develops a software classification system to verify an individual's identity based on their unique finger-vein pattern. 

\section{Background}
Vein recognition systems are based on the fact that vascular patterns can be captured in a cost-effective manner using infrared imaging. This is because the deoxygenated blood flowing through human veins absorb more infrared radiation than their surrounding tissue \cite{Pflug2012}. As a result, veins appear as dark regions in an infrared image of a human body part. Vein patterns are unique among individuals (including twins) \cite{Yanagawa2007} and different fingers of the same individual have different patterns as well \cite{Lu2013}. 
\par
The most common biometric currently used in identification today are fingerprints. However, other methods include authentication using the iris, face or even vocal characteristics. Advantages of vein recognition over these approaches are that:
\begin{itemize}
	\item Vein patterns cannot be forged as they are located beneath a person's skin. 
	\item Vein scanners do not require personal contact, hence they are more hygienic and suited for public areas \cite{Pflug2012}. 
	\item A vein-recognition system cannot be fooled by a severed finger or hand, as it does not have blood \cite{wilson2011vein}. 
	\item Unlike fingerprints, vein patterns are not affected by changes to the surface of the skin (such as wetness, cuts or bruises). Moreover, vein patterns remain relatively constant over an adult's lifetime \cite{Djerouni2014}. 
\end{itemize}

Table 1, compiled by Edgington, presents a comparison between biometric techniques used today \cite{edginton}. Advantages in its security are evident from the fact that vein patterns cannot be forged and are unique.

% Table generated by Excel2LaTeX from sheet 'Sheet1'
% Table generated by Excel2LaTeX from sheet 'Sheet1'
\begin{table}[htbp]
  \centering
  \caption{Comparison of different biometric methods}
    \begin{tabular}{rrrrrrrr}
    \toprule
    \multicolumn{1}{c}{\multirow{2}[0]{*}{Biometric}} & \multicolumn{2}{c}{Security} & \multicolumn{5}{c}{Practicality} \\
    \midrule
    \multicolumn{1}{c}{} & Anti-forgery & Accuracy & Speed & Enrollment & Convenience & Cost  & Size \\
    Fingerprint & Poor  & Average & Average & Poor  & Average & Good  & Good \\
    Iris  & Average & Good  & Average & Average & Poor  & Poor  & Poor \\
    Face  & Average & Poor  & Average & Average & Good  & Poor  & Poor \\
    Voice & Average & Poor  & Average & Average & Good  & Average & Average \\
    Finger Vein & Good  & Good  & Good  & Average & Average & Average & Average \\
    \bottomrule
    \end{tabular}%
  \label{tab:addlabel}%
\end{table}%

As shown above, vein recognition systems offer multiple advantages over other biometric technologies. As a result, the technology has recently been implemented by a number of international banks for client authentication at teller machines \cite{Vallabh2012}. These systems also tend to be cost effective and portable as the infrared light required for imaging can be obtained from a bank of correctly calibrated LEDs.

\section{Motivation}
 Vein recognition technology clearly shows great promise as a mode of personal authentication -- a technology which modern society is becoming increasingly dependent on. Services such as internet banking and automated access control to buildings require accurate authentication systems provide convenience to users. However, there are devastating consequences if intruders are able to access these systems. An accurate identification system, such as vein recognition, can not only be applied to these scenarios, but paves the way to a future where cash is no longer required and electronic payments can be made using just a secure personal identity. 

\section{Project Objectives}

The objective of this project is to show that vein recognition can be used to reliably identify individuals. This will be realised by the development of a software system which uses vein images as its input and correctly determines whether the subject of the image is in the training database or not. Multiple algorithms are to be implemented and compared in terms of accuracy and speed. Although various vein recognition algorithms have been developed by researchers around the world, they are difficult to compare directly as most of them have been tested on different datasets. This project will implement and compare some of these methods on the same dataset to determine their relative advantages and disadvantages.
\par
In order to produce a software system which has the accuracy and speed to be used in practical situations, the following key objectives have been identified:
\begin{itemize}
	\item A critical analysis of current literature and existing vein recognition methods.
	\item The development of all the components of a biometric authentication system as described in Lit Review barring the Sensor Module.
	\item The development of at least two distinct classification techniques.
	\item An evaluation of the developed system in terms of current biometric conventions.
	\item The best classification algorithm must have a Genuine Acceptance Rate (described in Lit Review) of at least 90\%.
	\item The implementation of the best algorithm on an embedded platform (\textit{Nvidia Jetson TK-1}) as a proof-of-concept for use in practical applications. 
	\item The system must be able to classify individuals within 1.5 seconds when running on the embedded platform. 
\end{itemize}
\section{Scope and Limitations}
This project develops the software required in a vein recognition system. The hardware needed to image the veins of an individual is not considered. Instead, the Shangdon University Homologous Multi-modal Traits Database (SDUMLA-HMT) \cite{Yin2011} is used as a source of finger-vein images whilst palm-vein samples were obtained from the Poznan University Vein Dataset \cite{Kabacinski2011}. A distinction is not made between finger-vein or palm-vein images as the problem of identifying and recognising the veins is essentially identical, barring the fact that the region of interest is different. The performance timing of the system only includes the execution time of the software, the time taken to image the subject's finger or hand is not considered. 
\par
The project is limited by the quality of the available images. Specifically, the non-uniform illumination of the subjects in both databases reduces the effectiveness of feature extraction techniques when processing vein images. Furthermore, the time constraint of twelve weeks on the project limits the number of different vein classification algorithms implemented and compared. 
\section{Software and Tools Used}
The development and testing of the software is primarily done using \textit{MATLAB}. The optimal recognition algorithm is then implemented on the \textit{Nvidia Jetson TK-1} embedded computer. The board is programmed in \textit{C++} using the Open Computer Vision (\textit{OpenCV}) library.   

\section{Report Outline}
This report is structured as follows:
\par
\textit{Chapter 2} is a literature review of existing pattern recognition techniques, followed by a focused discussion of how these methods have been applied to vein recognition.
\par
\textit{Chapter 3} details the various pre-processing techniques applied in the context of vein recognition. Three well-known methods, as well as one created by the author, are implemented and compared to each other.
\par
\textit{Chapter 4} describes a minutia-based method of feature extraction inspired by fingerprint recognition. The features are then classified using a Modified Hausdorff Distance metric and Nearest Neighbour classifier. An alternate method, using a novel centroid-based classifier, for matching these features is then discussed in \textit{Chapter 5}. Classification results are presented for both classifiers in their respective chapters.
\par
\textit{Chapter 6} introduces Fourier Descriptors as a means of extracting features based on the shape of the subject's finger. A method of combining previous features is also proposed.  
\par
\textit{Chapter 7} compares the results of the classification methods developed in the previous chapters. The performance of the algorithms on an embedded platform is also analysed.
\par
\textit{Chapters 8} and \textit{9} draw conclusions on the report and suggest recommendations for future research in the field of biometric vein recognition respectively.  



