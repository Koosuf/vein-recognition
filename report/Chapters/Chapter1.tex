\chapter{Introduction}

Personal identification is very important in the modern world. Biometrics, the statistical measurement of behavioural or physiological traits (CITE MIURA), is an effective way of dealing with this problem. Biometrics use unique characteristics of a person to identify them. Using biometric authentication technologies, users simply do not have to remember any passwords, since a unique part of the individual is the password. This also means that biometrics are much harder to fake as well.
\par
Common biometric technologies are fingerprint recognition and iris recognition. However, in this thesis, we explore vein recognition. Vein recognition identifies people according to the unique patterns of their veins. Vein patterns in various different parts of our body are unique \cite{jain2007handbook} - but it is most convenient to use the veins of our fingers or hands for recognition. In this thesis, we will work with finger-veins.
\par
The vein patterns in an individual's hand can be obtained by imaging it with Near-Infrared Radiation (NIR). The deoxygenated blood in the individuals veins will absorb the light whilst other parts of the hand will not. This results in an image where the dark regions correspond to veins. 
\par
Some benefits of finger-vein recognition go here.
\begin{itemize}
\item Can't forge it (under surface of skin)
\item Dead person's hand will not work (no blood)
\item Contact-less identification means more hygiene in public areas. 
\end{itemize}
\par
What databases I used
\par
Some obvious limitations (ie non-uniform illumination)
\par
Goal of this thesis is to show the finger-vein recognition can be used to reliably identify individuals. Furthermore, to show that a cost-effective and quick solution can also be developed. 

\section{Literature Review}

\begin{itemize}
\item Biometrics
	\begin{itemize}
		\item Something on Biometrics
		\item FAR, FRR, GAR and EER
		\item ROC Curves
	\end{itemize}
\item Image Processing
	\begin{itemize}
		\item Contrast and histogram equalisation
		\item Convolution using templates
		\item Morphology
		\begin{itemize}
			\item Intro
			\item Hit and Miss
			\item Erosion and Dilation
			\item Thinning and skeletonisation
		\end{itemize}
		\item Feature extraction
		\begin{itemize}
			\item What is a feature? - Can talk about what the dude said on MATLAB Central
			\item Histogram of gradients
			\item SURF	
		\end{itemize}
	\end{itemize}
\item Classifiers
	\begin{itemize}
			\item Nearest Neighbour
			\item Neural Network
	\end{itemize}
\end{itemize}

\subsection{A Subsection}

Donec urna leo, vulputate vitae porta eu, vehicula blandit libero. Phasellus eget massa et leo condimentum mollis. Nullam molestie, justo at pellentesque vulputate, sapien velit ornare diam, nec gravida lacus augue non diam. Integer mattis lacus id libero ultrices sit amet mollis neque molestie. Integer ut leo eget mi volutpat congue. Vivamus sodales, turpis id venenatis placerat, tellus purus adipiscing magna, eu aliquam nibh dolor id nibh. Pellentesque habitant morbi tristique senectus et netus et malesuada fames ac turpis egestas. Sed cursus convallis quam nec vehicula. Sed vulputate neque eget odio fringilla ac sodales urna feugiat.

\section{Another Section}

Phasellus nisi quam, volutpat non ullamcorper eget, congue fringilla leo. Cras et erat et nibh placerat commodo id ornare est. Nulla facilisi. Aenean pulvinar scelerisque eros eget interdum. Nunc pulvinar magna ut felis varius in hendrerit dolor accumsan. Nunc pellentesque magna quis magna bibendum non laoreet erat tincidunt. Nulla facilisi.

Duis eget massa sem, gravida interdum ipsum. Nulla nunc nisl, hendrerit sit amet commodo vel, varius id tellus. Lorem ipsum dolor sit amet, consectetur adipiscing elit. Nunc ac dolor est. Suspendisse ultrices tincidunt metus eget accumsan. Nullam facilisis, justo vitae convallis sollicitudin, eros augue malesuada metus, nec sagittis diam nibh ut sapien. Duis blandit lectus vitae lorem aliquam nec euismod nisi volutpat. Vestibulum ornare dictum tortor, at faucibus justo tempor non. Nulla facilisi. Cras non massa nunc, eget euismod purus. Nunc metus ipsum, euismod a consectetur vel, hendrerit nec nunc.